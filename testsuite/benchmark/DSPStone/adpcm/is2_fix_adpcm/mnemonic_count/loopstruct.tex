\documentstyle[12pt,german]{ert}
\begin{document}
\centerline{\bf Instruction Count vom Programm ``opt\_adpcm-v2.asm''}

\begin{itemize}
\item Aufz"ahlung aller Assembler Anweisungen, so oft wie sie im Code erscheinen.
\item Besondere Behandlung der Schleifen :\\
Es existieren 7 verschiedene Schleifen im Code, vier davon 
mit datenabh"angigen limits. Es wird der WORST CASE angenommen.Die
Anweisungen der Schleifeninhalte werden demnach mehrmals aufgez"ahlt.
\item Besondere Behandlung der Call Funktionen:\\
Da die Mehrheit der Code Funktionen zwei mal erscheinen, werden
sie als Call Funktionen geschrieben. Die entsprechende Assembler
Anweisungen dieser Funktionen sollen zweifach gez"ahlt werden.
\item Besondere Behandlung der MOVEs und TFR(transfer) Anweisungen:\\
Je nach Addressierungsart werden diese Anweisungen mit zwei oder mit
vier (bei indirekter Addressierung) Instruction Cycles ben"otigen.
Hierf"ur wurde eine besondere Aufz"ahlung durchgef"uhrt. Schlie\3lich 
werden die MOVES und TFR zu zwei verschiedenen Gruppen aufgez"ahlt.
\end{itemize}

\vspace{1mm}

Histogramme: 

\begin{itemize}
\item Verteilung der Mnemonics mit den oben genannten Methoden \\
``Mnemonic Count Histogram opt\_adpcm.asm''

\item Prozentuale Verteilung der Move \& Transfer Mnemonics \\
``Move \& Transfer analysis of opt\_adpcm.asm''
\end{itemize}

\begin{table}
\begin{tabular}{|c|c|c|} \hline
\multicolumn{3}{|c|}{Instruction Count} \\ \hline
Grp Number & Absolute Value & Percent Distribution \\ 
1 & 4993 & 59.15 \%\\
2 & 251 & 2.87 \% \\
3 & 214 & 2.45 \% \\
4 & 879 & 10.04 \% \\
5 & 0 & 0.0 \% \\
6 & 2231 & 25.49\% \\ \hline
\end{tabular}
\end{table}

\begin{table}
\begin{tabular}{|c|c|c|c|} \hline
\multicolumn{4}{|c|}{Move \& Transfer Analysis} \\ \hline
\multicolumn{2}{|c|}{2 Cycles Move \& Transfer}  &
\multicolumn{2}{|c|}{4 Cycles Move \& Transfer} \\ \hline
absolute & relative  & absolute & relative \\ 
3358  & 67.25 \%  & 1635 & 32.75 \% \\ \hline
\end{tabular}
\end{table}

\newpage

\centerline{ADPCM Project, Fix Version, First Loop}

{\small
\begin{verbatim}
 {	
    register int mag = MAG << 1;
    for (EXP = 0; mag >>= 1; EXP++)
                   ; 
  } 

192                                      L20
193                                      ; **************************************************
194                                      ;   {
195                                      ;     register int mag = MAG << 1;
196                                      ; **************************************************
197                                      ; **************************************************
198                                      ; 
199                                      ;     for (EXP = 0; mag >>= 1; EXP++)
200                                      ; **************************************************
201       P:0036 250000  [2 -      128]            move              #0,x1
202       P:0037 205800  [2 -      130]            move              (r0)+
203       P:0038 555000  [2 -      132]            move              b1,x:(r0)-
204       P:0039 205800  [2 -      134]            move              (r0)+
205       P:003A 56D000  [2 -      136]            move              x:(r0)-,a
206       P:003B 200032  [2 -      138]            asl     a
207       P:003C 218E00  [2 -      140]            move              a1,a
208       P:003D 218E00  [2 -      142]            move              a1,a
209       P:003E 200022  [2 -      144]            asr     a
210       P:003F 218E00  [2 -      146]            move              a1,a
211       P:0040 200003  [2 -      148]            tst     a
212       P:0041 0AF0AA  [6 -      154]            jeq     L18
                 00004B
213       P:0043 44F400  [4 -      158]            move              #>1,x0
                 000001
214                                      L8
215       P:0045 218E69  [2 -      160]            tfr     x1,b      a1,a
216       P:0046 200022  [2 -      162]            asr     a
217       P:0047 218E48  [2 -      164]            add     x0,b      a1,a
218       P:0048 21A503  [2 -      166]            tst     a         b1,x1
219       P:0049 0AF0A2  [6 -      172]            jne     L8
                 000045
220                                      L18
\end{verbatim}
}

\break 

\centerline{ADPCM Project, Fix Version, First Loop}


\begin{table}
\begin{tabular}{|c|c|c|c|c|c|} \hline
\multicolumn{5}{|c|}{LOOP 1} \\ \hline
Instruction &\multicolumn{2}{|c|}{Times} & \multicolumn{2}{|c|} {Cycles} \\ 
 & Number & Worst Case & Each &  Total \\ \hline
asr & 1 & 14 & 2 & 2 \\
add & 1 & 14 & 2 & 2 \\
tst & 1 & 14 & 2 & 2 \\
tfr & 1 & 14 & 2 & 2 \\
jne & 1 & 14 & 6 & 6 \\ \hline
\end{tabular}
\end{table}


Pro Schleifendurchgang : (2+2+2+2+6 = 14 instruction cycles)

{\bf WORST CASE: }

MAG ist 14 bits, Schleife wird 14 mal durchgef"uhrt, also 14*14 =
196 instruction cycles pro fmult() Aufruf. Diese Schleife wird optimal "ubersetzt.

\newpage

\centerline{ADPCM Project, Fix Version, Second Loop}

{\small
\begin{verbatim}
 {	
    register int mag = MAG << 1;
    for (EXP = 0; mag >>= 1; EXP++)
                   ; 
  } 

586                                      L38
587                                      ; **************************************************
588                                      ;   {
589                                      ;     register int mag = MAG << 1;
590                                      ; **************************************************
591                                      ; **************************************************
592                                      ; 
593                                      ;     for (EXP = 0; mag >>= 1; EXP++)
594                                      ; **************************************************
595       P:0182 381300  [2 -      942]            move              #19,n0
596       P:0183 270000  [2 -      944]            move              #0,y1
597       P:0184 57E800  [4 -      948]            move              x:(r0+n0),b
598       P:0185 380E3A  [2 -      950]            asl     b         #14,n0
599       P:0186 21AF00  [2 -      952]            move              b1,b
600       P:0187 21AF00  [2 -      954]            move              b1,b
601       P:0188 47682A  [4 -      958]            asr     b         y1,x:(r0+n0)
602       P:0189 381400  [2 -      960]            move              #20,n0
603       P:018A 21AF00  [2 -      962]            move              b1,b
604       P:018B 55680B  [4 -      966]            tst     b         b1,x:(r0+n0)
605       P:018C 0AF0AA  [6 -      972]            jeq     L112
                 00019C
606       P:018E 44F400  [4 -      976]            move              #>1,x0
                 000001
607                                      L41
608       P:0190 380E00  [2 -      978]            move              #14,n0
609       P:0191 57E800  [4 -      982]            move              x:(r0+n0),b
610       P:0192 200048  [2 -      984]            add     x0,b
611       P:0193 556800  [4 -      988]            move              b1,x:(r0+n0)
612       P:0194 381400  [2 -      990]            move              #20,n0
613       P:0195 57E800  [4 -      994]            move              x:(r0+n0),b
614       P:0196 21AF00  [2 -      996]            move              b1,b
615       P:0197 20002A  [2 -      998]            asr     b
616       P:0198 21AF00  [2 -     1000]            move              b1,b
617       P:0199 55680B  [4 -     1004]            tst     b         b1,x:(r0+n0)
618       P:019A 0AF0A2  [6 -     1010]            jne     L41
                 000190
619                                      L11
}
\end{verbatim}
}

\newpage

\centerline{ADPCM Project, Fix Version, Second Loop}

\begin{table}
\begin{tabular}{|c|c|c|c|c|c|} \hline
\multicolumn{5}{|c|}{LOOP 2} \\ \hline
Instruction &\multicolumn{2}{|c|}{Times} & \multicolumn{2}{|c|} {Cycles} \\ 
 & Number & Worst Case & Each &  Total \\ \hline
move & 4 & 14*4 = 56 & 2 & 8 \\
move & 3 & 14*3 = 42 & 4 & 12 \\
asr & 1 & 14 & 2 & 2 \\
add & 1 & 14 & 2 & 2 \\
tst & 1 & 14 & 4 & 4 \\
jne & 1 & 14 & 6 & 6 \\ \hline
\end{tabular}
\end{table}

Pro Schleifendurchgang : (8+12+2+2+4+6 = 34 instruction cycles)

{\bf WORST CASE: }

MAG ist 14 bits, Schleife wird 14 mal durchgef"uhrt, also 14*34 =
476 instruction cycles pro adpt\_predict() Aufruf.


Obwohl die 1. und die 2. Schleife den selben C Code haben, werden
beide unterschiedlich "ubersetzt aufgrund der register Deklarationen.
In der ersten Schleife wird die Variable EXP als erste Variable als register Variable
deklariert. Im 2. Fall erscheint die register Deklaration von EXP in
letzter Stelle und daher kann der selbe C code nicht optimal (wie im
1. Fall) "ubersetzt werden.

\newpage

\centerline{ADPCM Project, Fix Version, Third Loop}

{\small
\begin{verbatim}
 {	
    register int mag = MAG << 1;
    for (EXP = 0; mag >>= 1; EXP++)
                   ; 
  } 

1082                                     ;   {
1083                                     ;     register int mag = MAG << 1;
1084                                     ; **************************************************
1085                                     ; **************************************************
1086                                     ; 
1087                                     ;     for (EXP = 0; mag >>= 1; EXP++)
1088                                     ; **************************************************
1089      P:0337 380C00  [2 -     2098]            move              #12,n0
1090      P:0338 44F400  [4 -     2102]            move              #>16383,x0
                 003FFF
1091      P:033A 546800  [4 -     2106]            move              a1,x:(r0+n0)
1092      P:033B 381300  [2 -     2108]            move              #19,n0
1093      P:033C 47E400  [2 -     2110]            move              x:(r4),y1
1094      P:033D 205679  [2 -     2112]            tfr     y1,b      (r6)-
1095      P:033E 27004E  [2 -     2114]            and     x0,b      #0,y1
1096      P:033F 21AF00  [2 -     2116]            move              b1,b
1097      P:0340 55683A  [4 -     2120]            asl     b         b1,x:(r0+n0)
1098      P:0341 380E00  [2 -     2122]            move              #14,n0
1099      P:0342 21AF00  [2 -     2124]            move              b1,b
1100      P:0343 21AF00  [2 -     2126]            move              b1,b
1101      P:0344 47682A  [4 -     2130]            asr     b         y1,x:(r0+n0)
1102      P:0345 381400  [2 -     2132]            move              #20,n0
1103      P:0346 21AF00  [2 -     2134]            move              b1,b
1104      P:0347 55680B  [4 -     2138]            tst     b         b1,x:(r0+n0)
1105      P:0348 205600  [2 -     2140]            move              (r6)-
1106      P:0349 0AF0AA  [6 -     2146]            jeq     L111
                 000359
1107      P:034B 44F400  [4 -     2150]            move              #>1,x0
                 000001
1108                                     L82
1109      P:034D 380E00  [2 -     2152]            move              #14,n0
1110      P:034E 57E800  [4 -     2156]            move              x:(r0+n0),b
1111      P:034F 200048  [2 -     2158]            add     x0,b
1112      P:0350 556800  [4 -     2162]            move              b1,x:(r0+n0)
1113      P:0351 381400  [2 -     2164]            move              #20,n0
1114      P:0352 57E800  [4 -     2168]            move              x:(r0+n0),b
1115      P:0353 21AF00  [2 -     2170]            move              b1,b
1116      P:0354 20002A  [2 -     2172]            asr     b
1117      P:0355 21AF00  [2 -     2174]            move              b1,b
1118      P:0356 55680B  [4 -     2178]            tst     b         b1,x:(r0+n0)
1119      P:0357 0AF0A2  [6 -     2184]            jne     L82
                 00034D
1120                                     L111
\end{verbatim}
}

\begin{table}
\begin{tabular}{|c|c|c|c|c|c|} \hline
\multicolumn{5}{|c|}{LOOP 3} \\ \hline
Instruction &\multicolumn{2}{|c|}{Times} & \multicolumn{2}{|c|} {Cycles} \\ 
 & Number & Worst Case & Each &  Total \\ \hline
move & 4 & 14*4 = 56 & 2 & 8 \\
move & 3 & 14 * 3 =42&  4 & 12\\
asr & 1 & 14 & 2 & 2 \\
add & 1 & 14 & 2 & 2 \\
tst & 1 & 14 & 4 & 4 \\
jne & 1 & 14 & 6 & 6 \\ \hline
\end{tabular}
\end{table}

Pro Schleifendurchgang : (8+12+2+2+4+6 = 34 instruction cycles)

{\bf WORST CASE: }

MAG ist 14 bits, Schleife wird 14 mal durchgef"uhrt, also 14*34 =
476 instruction cycles pro adpt\_predict() Aufruf.

\newpage

\centerline{ADPCM Project, Fix Version, Fourth Loop}

{\small
\begin{verbatim}
for (EXP = 0; EXP < 6; EXP++)
  {
    DQS = MANT ? (BP[EXP] ? 65408 : 128) : 0;
    DQS += S->B[EXP] & (1 << 15)?(65536 - ((S->B[EXP] >> 8) + 65280)):
                                 (65536 - (S->B[EXP] >> 8));
    BP[EXP] = (S->B[EXP] + DQS) & 65535;
  }


1277                                     ; **************************************************
1278                                     ;   for (EXP = 0; EXP < 6; EXP++)
1279                                     ; **************************************************


1304      P:03EC 060680  [6 -     2570]            do      #6,L110
                 000442
1305                                     L95
1306                                     ; **************************************************
1307                                     ;   {
1308                                     ; #ifdef __LOOPCOUNT__ 
1309                                     ;     Loop4++;
1310                                     ; #endif
1311                                     ;     DQS = MANT ? (BP[EXP] ? 65408 : 128) : 0;
1312                                     ; **************************************************
1313      P:03EE 380F00  [2 -     2572]            move              #15,n0
1314      P:03EF 57E800  [4 -     2576]            move              x:(r0+n0),b
1315      P:03F0 20000B  [2 -     2578]            tst     b
1316      P:03F1 0AF0AA  [6 -     2584]            jeq     L89
                 000402
1317      P:03F3 56E200  [2 -     2586]            move              x:(r2),a
1318      P:03F4 200003  [2 -     2588]            tst     a
1319      P:03F5 0AF0AA  [6 -     2594]            jeq     L91
                 0003FB
1320      P:03F7 47F400  [4 -     2598]            move              #>65408,y1
                 00FF80
1321      P:03F9 0AF080  [6 -     2604]            jmp     L121
                 0003FD
1322                                     L91
1323      P:03FB 47F400  [4 -     2608]            move              #>128,y1
                 000080
1324                                     L121
1325      P:03FD 381400  [2 -     2610]            move              #20,n0
1326      P:03FE 476800  [4 -     2614]            move              y1,x:(r0+n0)
1327      P:03FF 47E800  [4 -     2618]            move              x:(r0+n0),y1
1328      P:0400 0AF080  [6 -     2624]            jmp     L122
                 000403
1329                                     L89
1330      P:0402 270000  [2 -     2626]            move              #0,y1
1331                                     L122
1332                                     ; **************************************************
1333                                     ;     DQS += S->B[EXP] & (1 << 15) ? (65536 - ((S->B[EXP] >> 8) + 65280)) :
1334                                     ;                                    (65536 - (S->B[EXP] >> 8));
1335                                     ; **************************************************
1336      P:0403 390F00  [2 -     2628]            move              #15,n1
1337      P:0404 381400  [2 -     2630]            move              #20,n0
1338      P:0405 228F00  [2 -     2632]            move              r4,b
1339      P:0406 476800  [4 -     2636]            move              y1,x:(r0+n0)
1340      P:0407 47E800  [4 -     2640]            move              x:(r0+n0),y1
1341      P:0408 380600  [2 -     2642]            move              #6,n0
1342      P:0409 476800  [4 -     2646]            move              y1,x:(r0+n0)
1343      P:040A 380E00  [2 -     2648]            move              #14,n0
1344      P:040B 47E800  [4 -     2652]            move              x:(r0+n0),y1
1345      P:040C 381378  [2 -     2654]            add     y1,b      #19,n0
1346      P:040D 21B100  [2 -     2656]            move              b1,r1
1347      P:040E 47E900  [4 -     2660]            move              x:(r1+n1),y1
1348      P:040F 476879  [4 -     2664]            tfr     y1,b      y1,x:(r0+n0)
1349      P:0410 47F400  [4 -     2668]            move              #>32768,y1
                 008000
1350      P:0412 20007E  [2 -     2670]            and     y1,b
1351      P:0413 21AF00  [2 -     2672]            move              b1,b
1352      P:0414 20000B  [2 -     2674]            tst     b
1353      P:0415 0AF0AA  [6 -     2680]            jeq     L93
                 000421
1354      P:0417 57E800  [4 -     2684]            move              x:(r0+n0),b
1355      P:0418 21AF00  [2 -     2686]            move              b1,b
1356      P:0419 0608A0  [4 -     2690]            rep     #8
1357      P:041A 20002A  [2 -     2692]            asr     b
1358      P:041B 21AF00  [2 -     2694]            move              b1,b
1359      P:041C 21A700  [2 -     2696]            move              b1,y1
1360      P:041D 57F400  [4 -     2700]            move              #>256,b
                 000100
1361      P:041F 0AF080  [6 -     2706]            jmp     L123
                 00042A
1362                                     L93
1363      P:0421 47E300  [2 -     2708]            move              x:(r3),y1
1364      P:0422 200079  [2 -     2710]            tfr     y1,b
1365      P:0423 21AF00  [2 -     2712]            move              b1,b
1366      P:0424 0608A0  [4 -     2716]            rep     #8
1367      P:0425 20002A  [2 -     2718]            asr     b
1368      P:0426 21AF00  [2 -     2720]            move              b1,b
1369      P:0427 21A700  [2 -     2722]            move              b1,y1
1370      P:0428 57F400  [4 -     2726]            move              #>65536,b
                 010000
1371                                     L123
1372                                     ; **************************************************
1373                                     ;     BP[EXP] = (S->B[EXP] + DQS) & 65535;
1374                                     ; **************************************************
1375                                     ; **************************************************
1376                                     ; **************************************************
1377      P:042A 38147C  [2 -     2728]            sub     y1,b      #20,n0
1378      P:042B 556800  [4 -     2732]            move              b1,x:(r0+n0)
1379      P:042C 380600  [2 -     2734]            move              #6,n0
1380      P:042D 57E800  [4 -     2738]            move              x:(r0+n0),b
1381      P:042E 381400  [2 -     2740]            move              #20,n0
1382      P:042F 47E800  [4 -     2744]            move              x:(r0+n0),y1
1383      P:0430 380678  [2 -     2746]            add     y1,b      #6,n0
1384      P:0431 556800  [4 -     2750]            move              b1,x:(r0+n0)
1385      P:0432 381400  [2 -     2752]            move              #20,n0
1386      P:0433 47E300  [2 -     2754]            move              x:(r3),y1
1387      P:0434 200078  [2 -     2756]            add     y1,b
1388      P:0435 20006E  [2 -     2758]            and     x1,b
1389      P:0436 21AF00  [2 -     2760]            move              b1,b
1390      P:0437 21A700  [2 -     2762]            move              b1,y1
1391      P:0438 556800  [4 -     2766]            move              b1,x:(r0+n0)
1392      P:0439 380E00  [2 -     2768]            move              #14,n0
1393      P:043A 226F00  [2 -     2770]            move              r3,b
1394      P:043B 476248  [2 -     2772]            add     x0,b      y1,x:(r2)
1395      P:043C 21B300  [2 -     2774]            move              b1,r3
1396      P:043D 224F00  [2 -     2776]            move              r2,b
1397      P:043E 200048  [2 -     2778]            add     x0,b
1398      P:043F 21B200  [2 -     2780]            move              b1,r2
1399      P:0440 57E800  [4 -     2784]            move              x:(r0+n0),b
1400      P:0441 200048  [2 -     2786]            add     x0,b
1401      P:0442 556800  [4 -     2790]            move              b1,x:(r0+n0)
1402                                     L110
\end{verbatim}
}
\newpage

\centerline{ADPCM Project, Fix Version, Fourth Loop}

\begin{table}
\begin{tabular}{|c|c|c|c|c|c|} \hline
\multicolumn{5}{|c|}{LOOP 4} \\ \hline
Instruction &\multicolumn{2}{|c|}{Times} & \multicolumn{2}{|c|} {Cycles} \\ 
 & Number & Worst Case & Each &  Total \\ \hline
move & 29 & 6 * 29 = 174 & 2 & 58 \\
move & 21& 6 * 21 = 126 & 4 & 84 \\
asr & 2 & 6*2 = 12 & 2*8 & 32 \\
add & 6 & 6*6 = 36 & 2 & 12 \\
tst & 3 & 6*3 = 18 & 2 & 6 \\
tfr & 1 & 6 & 2 & 2 \\
tfr & 1 & 6 & 4 & 4\\ 
rep & 2 & 6*2=12 & 4 & 8 \\
sub & 1 & 6*2=12&2 & 2 \\
and & 2 & 6*2 = 12 & 2 & 4 \\
do & 1 & 1 & 6 & 6 \\
jmp & 3 & 6*3=18 & 6 & 18 \\ 
jeq & 3 & 6*3= 18& 6  & 18\\\hline
\end{tabular}
\end{table}

{\bf WORST CASE: }

Die Schleife l"auft mit einem "au\3eren Schleifenwert von 6 (constant)

Im WORST CASE werden in der Schleife (insgesamt) 6 +  6 * 194  = 1170 instruction
cycles ben"otigt.

\newpage 
\centerline{ADPCM Project, Fix Version, Fifth Loop}

{\small
\begin{verbatim}
 if (S->T)
    for (EXP = 0; EXP < 6; EXP++)
      BP[EXP] = 0 ; 
1406                                     ;   /* TRIGB */
1407                                     ;   if (S->T)
1408                                     ; **************************************************
1409      P:0443 205C00  [2 -     2792]            move              (r4)+
1410      P:0444 56D400  [2 -     2794]            move              x:(r4)-,a
1411      P:0445 200003  [2 -     2796]            tst     a
1412      P:0446 0AF0AA  [6 -     2802]            jeq     L96
                 00044E
1413                                     ; **************************************************
1414                                     ;     for (EXP = 0; EXP < 6; EXP++)
1415                                     ; **************************************************
1416      P:0448 221100  [2 -     2804]            move              r0,r1
1417      P:0449 240000  [2 -     2806]            move              #0,x0
1418      P:044A 060680  [6 -     2812]            do      #6,L108
                 00044C
1419                                     L100
1420                                     ; **************************************************
1421                                     ; #ifdef __LOOPCOUNT__ 
1422                                     ;     {
1423                                     ;       Loop5++;
1424                                     ;       BP[EXP] = 0 ;
1425                                     ;     }
1426                                     ; #else
1427                                     ;       BP[EXP] = 0 ;
1428                                     ; **************************************************
1429                                     ; **************************************************
1430                                     ; **************************************************
1431      P:044C 445900  [2 -     2814]            move              x0,x:(r1)+
1432                                     L108
1433      P:044D 000000  [2 -     2816]            nop
1434                                     L9 
\end{verbatim}
}

\begin{table}
\begin{tabular}{|c|c|c|c|c|c|} \hline
\multicolumn{5}{|c|}{LOOP 5} \\ \hline
Instruction &\multicolumn{2}{|c|}{Times} & \multicolumn{2}{|c|} {Cycles} \\ 
 & Number & Worst Case & Each &  Total \\ \hline
move & 1 & 6 & 2 & 2 \\
do & 1 & 1 & 6 & 6 \\  \hline
\end{tabular}
\end{table}


Die Schleife l"auft mit einem "au\3eren Schleifenwert von 6 (constant)

Es werden im WORST CASE ( (S-$>$T) is TRUE) in der Schleife (insgesamt) 
6 + 2*6  = 18 instruction cycles ben"otigt. Dieser Loop wird optimal "ubersetzt.

\newpage

\centerline{ADPCM Project, Fix Version, Sixth Loop}

{\small
\begin{verbatim}
  for (EXP = 0; EXP < 6; EXP++)
    S->B[EXP] = BP[EXP];
1588                                     ; **************************************************
1589                                     ;   for (EXP = 0; EXP < 6; EXP++)
1590                                     ; **************************************************
1607      P:04CD 060680  [6 -     3156]            do      #6,L106
                 0004D5
1608                                     L104
1609                                     ; **************************************************
1610                                     ; #ifdef __LOOPCOUNT__
1611                                     ;   {
1612                                     ;     Loop6++;
1613                                     ;     S->B[EXP] = BP[EXP];
1614                                     ;   }
1615                                     ; #else
1616                                     ;     S->B[EXP] = BP[EXP];
1617                                     ; **************************************************
1618                                     ; **************************************************
1619                                     ; **************************************************
1620      P:04CF 381400  [2 -     3158]            move              #20,n0
1621      P:04D0 224F00  [2 -     3160]            move              r2,b
1622      P:04D1 47D948  [2 -     3162]            add     x0,b      x:(r1)+,y1
1623      P:04D2 476800  [4 -     3166]            move              y1,x:(r0+n0)
1624      P:04D3 476200  [2 -     3168]            move              y1,x:(r2)
1625      P:04D4 21B200  [2 -     3170]            move              b1,r2
1626      P:04D5 000000  [2 -     3172]            nop
1627                                     L106
\end{verbatim}
}

\begin{table}
\begin{tabular}{|c|c|c|c|c|} \hline
\multicolumn{5}{|c|}{LOOP 6} \\ \hline
Instruction &\multicolumn{2}{|c|}{Times} & \multicolumn{2}{|c|} {Cycles} \\ 
 & Number & Worst Case & Each &  Total \\ \hline
move & 4 & 6*4 = 24 & 2 & 8 \\
move & 1& 6*1 = 6 & 4 & 4 \\
add & 1 & 6*1 = 6& 2 & 2 \\
do & 1 & 1& 6 & 6 \\
nop & 1 & 6*1 = 6 & 2 & 2\\ \hline
\end{tabular}
\end{table}

{\bf WORST CASE: }

Die Schleife l"auft mit einem "au\3eren Schleifenwert von 6 (constant)

In allen F"allen wird in der Schleife (insgesamt) 6 + (6 * 16) = 102 instruction
cycles ben"otigt.
\newpage

\centerline{ADPCM Project, Fix Version, Seventh Loop}

{\small
\begin{verbatim}
 {	
   register int dqm = DQM;
   for (EXP = 1; dqm >>= 1; EXP++)
         ; 
 }
3026                                     L183
3027                                     ; **************************************************
3028                                     ;       {
3029                                     ;       register int dqm = DQM;
3030                                     ; **************************************************
3031                                     ; **************************************************
3032                                     ; 
3033                                     ;       for (EXP = 1; dqm >>= 1; EXP++)
3034                                     ; **************************************************
3035      P:088B 46F400  [4 -     5410]            move              #>1,y0
                 000001
3036      P:088D 218500  [2 -     5412]            move              a1,x1
3037      P:088E 0AF080  [6 -     5418]            jmp     L229
                 000894
3038                                     L186
3039      P:0890 47F459  [4 -     5422]            tfr     y0,b      #>1,y1
                 000001
3040      P:0892 200078  [2 -     5424]            add     y1,b
3041      P:0893 21A600  [2 -     5426]            move              b1,y0
3042                                     L229
3043      P:0894 218E00  [2 -     5428]            move              a1,a
3044      P:0895 200022  [2 -     5430]            asr     a
3045      P:0896 218E00  [2 -     5432]            move              a1,a
3046      P:0897 200003  [2 -     5434]            tst     a
3047      P:0898 0AF0A2  [6 -     5440]            jne     L186
\end{verbatim}
}

\begin{table}
\begin{tabular}{|c|c|c|c|c|c|} \hline
\multicolumn{5}{|c|}{LOOP 7} \\ \hline
Instruction &\multicolumn{2}{|c|}{Times} & \multicolumn{2}{|c|} {Cycles} \\ 
 & Number & Worst Case & Each &  Total \\ \hline
move & 3 & 14*3= 42 & 2 & 6 \\
asr & 1 & 14 & 2 & 2\\
add & 1 & 14 &2 & 2 \\
tst & 1 & 14 & 2 & 2 \\
tfr & 1 & 14 & 4 & 4 \\ 
jne & 1 & 14 & 6 & 6 \\ \hline
\end{tabular}
\end{table}


Pro Schleifendurchgang : (6+2+2+2+4+6 = 22 instruction cycles)

{\bf WORST CASE: }

DQM ist maximal 14 bits lang, Schleife wird dann 14 mal durchgef"uhrt, also 14*22 =
308 instruction cycles pro fmult() Aufruf.

\newpage

%\begin{table}
%%\begin{tabular}{|c|c|c|c|c|} \hline
%\multicolumn{4}{|c|}{fmult function} \\ \hline
%Instruction &Times & \multicolumn{2}{|c|} {Cycles} \\ 
% & & Each & Total \\ \hline
%move & 72 & 2 & 144 \\
%move & 25 & 4 & 100 \\
%add & 4 & 2 & 8 \\
%tst & 13 & 2 & 26 \\
%jeq & 9 & 6 & 54 \\
%jmp & 6 & 6 & 36 \\
%asr & 12 & 2 & 24 \\
%asl & 4 & 2 & 8 \\
%tfr & 5 & 2 & 10 \\
%tfr & 2 & 4 & 8 \\
%jge & 2 & 6 & 12 \\
%neg & 2 & 2 & 4 \\
%rep & 7 & 4 & 28 \\
%sub & 4 & 2 & 8\\
%and & 6 & 2 & 12 \\
%eor & 1 & 2 & 2 \\
%mac & 1 & 2 & 2 \\
%jne & 1 & 6 & 6\\
%rts & 1 & 4 & 4 \\ \hline
%\end{tabular}
%\end{table}

%Pro Schleifendurchgang : (6+2+2+2+4+6 = 22 instruction cycles)
%{\bf WORST CASE: }
%DQM ist maximal 14 bits lang, Schleife wird dann 14 mal durchgef"uhrt, also 14*22 =
%308 instruction cycles pro fmult() Aufruf.


\newpage



\end{document}

\begin{verbatim}
ASSEMBLER GROUPS   DSP56K (See also cap 7.3, pp.7-18 ss.)
================

GROUP 1	MOVE INSTRUCTIONS
-------
LUA	
MOVE	
MOVEC	
MOVEM	
MOVEP	
TCC
TCS
THS
TLO
TEC
TEQ
TES
TGE
TGT
TLC
TLE
TLS
TLT
TMI
TNE
TNR
TPL
TNN
TFR


GROUP 2	LOGICAL INSTRUCTIONS
-------
AND	
ANDI	
EOR	
LSL	
LSR	
NOT
OR	
ORI
ROL
ROR



GROUP 3 LOOP AND PROGRAM CONTROL INSTRUCTIONS
-------
DO	
ENDDO
ILLEGAL
NOP
REP
RESET
RTI
RTS
STOP
SWI
WAIT


GROUP 4 CONTROL INSTRUCTIONS
-------
JMP
JSR
JCC
JCS
JHS 
JLO
JEC
JEQ
JGE
JGT
JLC
JLE
JLS
JLT
JMI
JNE
JNR
JPL
JNN
JSCC
JSCS
JSHS
JSLO
JSEC
JSEQ
JSES
JSGE
JSGT
JSLC
JSLS
JSLT
JSMI
LSNE
JSNR
JSPL
JSNN
JCLR
JSCLR
JSET
JSSET

GROUP 5 BIT OPERATIONS
-------
BCLR
BSET
BCHG
BTST

GROUP 6 ARITHMETIC INSTRUCTIONS
-------
ABS
ADC
ADD
ADDL
ADDR
ASL
ASR
CLR
CMP
CMPM
DIV
MAC
MACR
MPY
MPYR
NEG
NORM
RND
SBC
SUB
SUBL
SUBR
TST

\end{verbatim}


\end{document} 